\chapter{Conclusion and Future Work}
\label{chap:Conclusion}

We proposed that the condition number distribution in detection region can be interpreted as the accuracy function. Through this accuracy function we can easily compute the accuracy in the tracking system at given camera pose. Also we already verified the it's feasibility and robustness through comparing the rotational error and translational. Through comparison of different paths computed with regular \texttt{A*} and artificial potential fields method based on condition number distribution, we know that one possible most accurate path in the detection region of the marker is like a curve, the robot should move firstly towards the marker and then outwards the marker.

In the future work maybe we can find out a better solution for normalization of the image points. We also need to modify \texttt{A*} search algorithm, which means the heuristic function should based on condition number distribution on the grid. For artificial potential fields method in each iteration we need to expand the search region from 8 adjacent nodes to 24 adjacent nodes. We also need to increase the resolution of the grid, e.g. 0.05m or 0.01m. Then We should compare the performance of different search algorithms in following ways:
\begin{enumerate}
\item Completeness: does the algorithm find out the solution when there exists one?
\item Optimality: is the solution the best one of all possibility in terms of path cost?
\item Time complexity: how long does it take to find a solution?
\item Space complexity: how much memory is needed to perform the search algorithm?
\end{enumerate}

